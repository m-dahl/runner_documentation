% Flowcharting techniques for easy maintenance
% Author: Brent Longborough
\documentclass[x11names]{article}
\usepackage{tikz}
\usetikzlibrary{shapes,arrows,chains}
\begin{document}
\section*{sp threader runner.rs}
% =================================================
% Set up a few colours
\colorlet{lcfree}{Green3}
\colorlet{lcnorm}{Blue3}
\colorlet{lccong}{Red3}
% -------------------------------------------------
% Set up a new layer for the debugging marks, and make sure it is on
% top
\pgfdeclarelayer{marx}
\pgfsetlayers{main,marx}
% A macro for marking coordinates (specific to the coordinate naming
% scheme used here). Swap the following 2 definitions to deactivate
% marks.
\providecommand{\cmark}[2][]{%
  \begin{pgfonlayer}{marx}
    \node [nmark] at (c#2#1) {#2};
  \end{pgfonlayer}{marx}
  }
\providecommand{\cmark}[2][]{\relax}
% -------------------------------------------------
% Start the picture
\begin{tikzpicture}[%
    >=triangle 60,              % Nice arrows; your taste may be different
    start chain=going below,    % General flow is top-to-bottom
    node distance=6mm and 60mm, % Global setup of box spacing
    every join/.style={norm},   % Default linetype for connecting boxes
    ]
% -------------------------------------------------
% A few box styles
% <on chain> *and* <on grid> reduce the need for manual relative
% positioning of nodes
\tikzset{
  base/.style={draw, on chain, on grid, align=center, minimum height=4ex},
  proc/.style={base, rectangle, text width=8em},
  test/.style={base, diamond, aspect=2, text width=5em},
  term/.style={proc, rounded corners},
  % coord node style is used for placing corners of connecting lines
  coord/.style={coordinate, on chain, on grid, node distance=6mm and 25mm},
  % nmark node style is used for coordinate debugging marks
  nmark/.style={draw, cyan, circle, font={\sffamily\bfseries}},
  % -------------------------------------------------
  % Connector line styles for different parts of the diagram
  norm/.style={->, draw, lcnorm},
  free/.style={->, draw, lcfree},
  cong/.style={->, draw, lccong},
  it/.style={font={\small\itshape}}
}
% -------------------------------------------------
% Start by placing the nodes
\node [proc, densely dotted, it] (p0) {Runner input};
% Use join to connect a node to the previous one
\node [term, join]      {Spec change? $\rightarrow$ Reset cache};
\node [term, join] (p1) {Manual state change? $\rightarrow$ Check disabled ops};
\node [proc, join]      {Tick transitions};
\node [test, join] (t1) {Safe state?};
% No join for exits from test nodes - connections have more complex
% requirements
% We continue until all the blocks are positioned
\node [proc] (p2) {Update current goals};
\node [proc, join] (p3) {For high then low level:};
\node [test, join] (t2) {Goal changed?};
\node [test] (t3) {Can reach goal?};
\node [proc] (t4) {Level done};
% We position the next block explicitly as the first block in the
% second column.  The chain 'comes along with us'. The distance
% between columns has already been defined, so we don't need to
% specify it.
\node [proc, right=of p1] (p4) {Lift offending specs};
\node [proc, join] {Find plan to state where spec holds};
\node [proc, join] (p5) {Suggest actions to user};

% Replan
\node [proc, right=of p3] (t6) {Replan};
% Some more nodes specifically positioned (we could have avoided this,
% but try it and you'll see the result is ugly).
\node [test, join] (t7) [right=of t2] {Success?};
\node [proc, join, right=of t3] (p8) {Packing heuristic};
\node [proc, right=45mm of t7] (p9) {Mark offending operation as disabled};
\node [term, join] (p10) {Forget goals to trigger replan};
% -------------------------------------------------
% Now we place the coordinate nodes for the connectors with angles, or
% with annotations. We also mark them for debugging.
\node [coord, right=of t1] (c1)  {}; % \cmark{1}
\node [coord, right=of t2] (c2)  {}; % \cmark{2}
\node [coord, right=of t3] (c3)  {}; % \cmark{3}
\node [coord, right=of t6] (c6)  {}; % \cmark{6}
\node [coord, right=of t7] (c7)  {}; % \cmark{7}
\node [coord, left=of t4]  (c4)  {}; % \cmark{4}
\node [coord, right=of t4] (c4r) {}; % \cmark[r]{4}
\node [coord, left=of t6]  (c5)  {}; % \cmark{5}
% -------------------------------------------------
% A couple of boxes have annotations
\node [above=0mm of p4, it] {We are in a forbidden state w.r.t. specs};
\node [above=0mm of p8, it] {Heuristic depends on planning level};
% -------------------------------------------------
% All the other connections come out of tests and need annotating
% First, the straight north-south connections. In each case, we first
% draw a path with a (consistently positioned) annotation node, then
% we draw the arrow itself.
\path (t1.south) to node [near start, xshift=1em] {$y$} (p2);
  \draw [*->,lcnorm] (t1.south) -- (p2);
\path (t2.south) to node [near start, xshift=1em] {$n$} (t3);
  \draw [*->,lcnorm] (t2.south) -- (t3);
\path (t3.south) to node [near start, xshift=1em] {$y$} (t4);
  \draw [*->,lcnorm] (t3.south) -- (t4);
% \path (t5.south) to node [near start, xshift=1em] {$y$} (t6);
%   \draw [*->,lcfree] (t5.south) -- (t6);
% \path (t6.south) to node [near start, xshift=1em] {$y$} (t7);
%   \draw [*->,lcfree] (t6.south) -- (t7);
% -------------------------------------------------
% Now the straight east-west connections. To provide consistent
% positioning of the test exit annotations, we have positioned
% coordinates for the vertical part of the connectors. The annotation
% text is positioned on a path to the coordinate, and then the whole
% connector is drawn to its destination box.

% -------------------------------------------------
% Finally, the twisty connectors. Again, we place the annotation
% first, then draw the connector
\path (t1.east) to node [near start, yshift=1em] {$n$} (c1);
  \draw [o->,lcnorm] (t1.east) -- (c1) |- (p4);

  \draw [*->,lcnorm] (t4.west) -- (c4) |- (p3);

  \draw [->,lcnorm] (p10) |- (t4.east);


\path (t7.east) to node [near start, yshift=1em] {$n$} (c7);
  \draw [o->,lcnorm] (t7.east) -- (c7) |- (p9);

\draw [->,lcnorm] (p8.south) |- (t4);

\path (t2.east) to node [yshift=-1em] {$y$} (c2);
\draw [*->,lcnorm] (t2.east) -- (c2) |- (t6);

\path (t3.east) to node [yshift=-1em] {$n$} (c3);
\draw [o->,lcnorm] (t3.east) -- (c3) |- (t6);

% -------------------------------------------------
% A last flourish which breaks all the rules
\draw [->,dotted, thick, shorten >=1mm]
  (t4.south) -- ++(0mm,-5mm)  -- ++(-40mm,0)
  |- node [black, near end, yshift=1em, it]
    {External input} (p0);
% -------------------------------------------------
  \end{tikzpicture}

% =================================================
\end{document}

%%% Local Variables:
%%% mode: latex
%%% TeX-master: t
%%% End:
